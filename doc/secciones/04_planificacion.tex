\chapter{Planificación}

\section{Metodología utilizada}

Para el desarrollo de este proyecto se ha elegido como metodología el \textbf{Desarrollo Ágil} \cite{tas}.
En mi opinión, este tipo de enfoque se acerca más al que desarrollan las empresas en la actualidad, donde se
definen sprints en los que los requisitos y soluciones van evolucionando según la necesidad del proyecto.
No obstante cada iteración deberá tener su planificación, análisis de requisitos, diseño, etc.

\section{Temporización}

La temporización general de este proyecto va a estar dividida en dos: construcción del Web Service para la
gestión de los datos del aplicativo y la creación de un entorno Front-End para los usuarios. A su vez estos dos
objetivos principales estarán divididos en diferentes hitos para realizar un desarrollo progresivo.

\vspace{0.5em}
\begin{itemize}
    \item \textbf{Web Service}:
        \begin{itemize}
            \item \textbf{Gestión de citas previas}: En este primer hito se completará toda la funcionalidad referente
            a la gestión de cita previa. Contendrá todo lo necesario para que desde la aplicación Webs se puedan
            realizar operaciones como reserva de citas por parte de usuarios; obtención de histórico de citas ya
            sea del negocio, usuarios o trabajadores; cancelación de citas; así como todas las comprobaciones y
            controles para la gestión de estas citas. De la mano de este hito se completará también la gestión
            de inicio de sesión de usuario y la encriptación de datos privados en la base de datos del aplicativo.

            \item \textbf{Gestión de venta de productos online}: Una vez finalizadas las dos etapas anteriores
            pasaremos a desarrollar otro de los puntos principales del problema, la gestión de venta online de
            productos. En este hito se deberá implementar la gestión de los pedidos que lleguen desde la aplicación
            Front-End de los usuarios de las Webs.

            \item \textbf{Administración Back-End}: Por último se va a añadir para el Back-End desarrollado una vista
            de administración para poder acceder a todos los datos registrados por cada negocio. Además se podrán
            añadir funcionalidades para los trabajadores o un panel de estadísticas donde ver información relevante.
        \end{itemize}
    \item \textbf{Aplicación Web/Móvil}:
    \begin{itemize}
                \item \textbf{Esqueleto y Home}: El primer hito correspondiente al desarrollo Front-End del aplicativo
                será la construcción del esqueleto Web (header, barra de navegación, footer, etc.) Junto con la Home
                con información y presentación del negocio.

                \item \textbf{Interfaz para la reserva de citas}: Una vez cumplido el primer hito pasaremos al
                desarrollo de lo referente a las citas previas. Se creará una vista donde ver las citas disponibles del
                negocio y poder reservarlas como usuario autenticado, además de una vista individual donde ver
                un histórico de citas reservadas y próximas citas.

                \item \textbf{Interfaz para las compras online de productos}: Por último finalizaremos este trabajo con
                el desarrollo de la interfaz para la tienda online. Se creará una vista donde se muestren los productos
                del negocio con los filtros correspondientes para los productos. Se podrán añadir productos al carrito
                de compra y realizar el pago de los productos deseados para realizar los pedidos.

            \end{itemize}
\end{itemize}
\subsection{Sprints definidos}

Además de la organización para el desarrollo del sistema, también se debe organizar el tiempo a invertir para el análisis previo necesario y el diseño de base de datos o interfaces de la aplicación. Una vez tengamos todos los puntos estudiados y definidos podremos comenzar con el desarrollo en programación. Podemos ver la organización prevista en la siguiente imagen:


\begin{figure}[H]
  \centering
  \includegraphics[scale=0.27]{images/Sprint_Prevision.png}
  \caption{Previsión de sprint a desarrollar}
  \label{}
\end{figure}

Como se indica, el tiempo transcurrido desde que se comienza este proyecto hasta que se pasa a la parte de desarrollo web es de aproximadamente mes y medio. Esto es debido a la cantidad de información y casuísticas que debemos analizar. Además el análisis previo es muy importante a la hora del diseño de la base de datos, ya que a partir de este obtendremos toda la información necesaria de almacenar y las relaciones que tendremos entre nuestras entidades.

La programación ocupará también alrededor de un mes y medio de trabajo. Con todo el análisis y diseño ya realizado no debería llevar un tiempo muy prologado su implementación, aunque siempre podemos encontrarnos problemas de integración con herramientas externas que utilicemos.

Por último tendremos dos semanas para el despliegue en producción y pruebas para comprobar que todo funciona perfectamente.

\section{Presupuesto}

Suponiendo que el proyecto se realizara bajo una contratación para su desarrollo, vamos a proceder a hacer un desglose del presupuesto que sería necesario.

En primer lugar, haremos un cálculo del salario mensual de un programador junior. Observando los diferentes portales de ofertas de trabajo como pueden ser InfoJobs o LinkedIn vemos que los salarios netos anuales rondan entre los 16.000 y 24.000 euros brutos. Supondremos un sueldo medio de entre el rango dado, es decir, 20.000€, que al mes supondrán un gasto de 1.666 € aproximadamente.\\

En segundo lugar, calcularemos el gasto añadido de las licencias de software requeridas. Las únicas licencias que supondrán un gasto extra en el proyecto serán las de los IDEs utilizados, debido a que para los repositorios de los proyectos haremos uso de GNU General Public License v3.0. Al usar varios IDEs para el desarrollo de este proyecto como es PHPStorm (para la parte de Back-End), WebStorm (para Front-End) y DataGrip (para el acceso a base de datos) usaremos el paquete en la que se proporcionan todas las herramientas de JetBrains con un precio de 301,29€, el primer año. Por la duración del proyecto (4 meses) realizaremos una regla de tres para calcular cual sería el importe a pagar en estos meses de desarrollo, siendo 100,43€.\\ 

Por último añadiremos al presupuesto el hardware utilizado. En este caso tendremos un ordenador portátil Intel® Core™ de 7.ª generación y una tarjeta gráfica independiente NVIDIA GTX 1050 por 700€. Supondremos que se realizará una financiación de este equipo en los meses de trabajo por lo que supondrá un gasto mensual de 175€.

\subsection{Factura total}

Realizaremos finalmente el cálculo total (aproximado) de los costes por mes y total del proyecto con la duración que se estimó en los puntos anteriores:

\begin{table}[H]
\begin{tabular}{lll}
\hline
\rowcolor[HTML]{EFEFEF}
\multicolumn{1}{|l|}{\cellcolor[HTML]{EFEFEF}\textbf{Tipo de Gasto}} & \multicolumn{1}{l|}{\cellcolor[HTML]{EFEFEF}
    \textbf{Mensual}
} & \multicolumn{1}{l|}{\cellcolor[HTML]{EFEFEF}
    \textbf{Total (4 meses)}
} \\ \hline
\multicolumn{1}{|l|}{\textbf{Salario}} & \multicolumn{1}{l|}{
    1.666€
}  & \multicolumn{1}{l|}{
    6.664€
} \\ \hline
\multicolumn{1}{|l|}{\textbf{Software}} & \multicolumn{1}{l|}{
    25,11€
}  & \multicolumn{1}{l|}{
    100,43€
} \\ \hline
\multicolumn{1}{|l|}{\textbf{Hardware}} & \multicolumn{1}{l|}{
    175€
}  & \multicolumn{1}{l|}{
    700€
} \\ \hline

\multicolumn{3}{l}{} \\ \hline
\rowcolor[HTML]{EFEFEF}
\multicolumn{1}{|l|}{\cellcolor[HTML]{EFEFEF}\textbf{Gasto total}} & \multicolumn{1}{l|}{
    1.866,11€
}  & \multicolumn{1}{l|}{
    7.464,44€
} \\ \hline

\end{tabular}
\end{table}