\section{Descripción del problema}

Dado un caso real, donde a una barbería normalmente acudía la gente sin previo aviso para esperar a
poder ser atendida, tuvo que adaptarse a las nuevas normativas y habilitar un teléfono para la reserva de citas.

Se plantea a partir de esta situación, la creación de una aplicación web/móvil que permita la gestión de las citas
automáticamente, sin necesidad de que el trabajador se distraiga de su labor. Facilitando también a los usuarios la
consulta de horarios y disponibilidad del mismo, permitiendo elegir la fecha deseada con unas simples acciones,
además de tener un registro de las diferentes reservas.

\section{Estado del arte}

El software libre y sus licencias \cite{gplv3} ha permitido llevar a cabo una expansión del
aprendizaje de la informática sin precedentes.

\subsection{Aplicaciones actuales}

En la actualidad existen diversas páginas para la reserva de citas en distintos negocios de cita previa:

\vspace{-0.5em}
\begin{itemize}
    \item \textbf{miCita}: Aplicación móvil desarrollada con el propósito de gestionar el apartado de citas de un
    negocio o empresa. En ella se rellenarán los datos del propio negocio con un asistente de configuración donde
    se completará información como los servicios ofrecidos y horarios. Los clientes podrán reservar su cita y
    observar el calendario en tiempo real desde la aplicación. Para acceder a estas funcionalidades como cliente
    se deberá abonar una cuota mensual de \textbf{19,90\euro/mes (IVA no incluido)}.\cite{micita}

    \item \textbf{Reservio}: Aplicación web para la gestión de reservas o citas en un negocio.
    Con un calendario para las reservas se podrá acceder como trabajador o como cliente permitiendo la reserva online.
    Además tendrá la posibilidad de gestionar los perfiles de los clientes  y estos podrán ser recompensados
    con pases de fidelidad. Esta aplicación también incluye asistencia de negocio con estadísticas y
    recordatorios vía SMS.\cite{reservio}

    \item \textbf{Shopify}: Shopify es una plataforma online para montar webs de venta online. Con Shopify se puede
    montar webs con diseños predefinidos y despreocuparte de la gestión de marketing, transacciones online
    y envíos. Para acceder a este contenido y poder montar un e-commerce Shopify ofrece cuotas mensuales que parten de
    \textbf{29\$/mes}\cite{shopify}
\end{itemize}

\subsection{Crítica al estado del arte}

Las aplicaciones mencionadas en el apartado anterior son los subconjuntos necesarios para la creación de este proyecto.
Las dos primeras incluyen la gestión de citas, y usuario (en el caso de \textit{Reservio}). La tercera web constituye
la parte de la venta online, donde se gestiona todo lo relacionado con pedidos, pagos, envíos, etc.:

\vspace{-0.5em}

\begin{itemize}
    \item \textbf{miCita}: un punto negativo al usar una aplicación como esta es que no tienes una interfaz propia
    que te diferencie de otras tiendas que usen este mismo medio.

    \item \textbf{Reservio}: se trata de una aplicación de gestión de negocio de citas gratuita, cosa positiva a
    valorar, que permite la reserva de citas por partes de los clientes. Sin embargo con este aplicativo no se
    podría cubrir la venta online de un negocio, ya que no da soporte para ello.

    \item \textbf{Shopify}: al contrario que la anterior, Shopify se trata de una herramienta de creación de webs
    para venta de productos online. Al ser solo la construcción de la tienda online no tendríamos la posibilidad de
    tener unificada en una misma aplicación nuestra gestión de citas y ventas online.
\end{itemize}

\subsection{Propuesta}
Teniendo en cuenta la situación actual sobre el estado del arte en lo referente a aplicaciones para
gestión de negocios de cita previa, se propone crear un sistema genérico que permita la administración
de estos. La principal finalidad del proyecto es conseguir un sistema sólido y unificado, en el que se
puedan integrar todos los negocios que se deseen (de características similares) de una manera sencilla.

No obstante al hacer una separación entre Front-End y Back-End será posible hacer única cada web si se
desea aunque internamente estén montadas con la misma estructura de datos.

Además de la gestión de citas se incluirán las funcionalidades necesarias para la implementación de una
tienda online en el que cada negocio podrá vender sus productos sin necesidad de atarse únicamente
a sus clientes locales.

Para la parte del Back-End se creará una vista de administración con roles de usuario:

\vspace{-0.5em}
\begin{itemize}
    \item Cada negocio deberá tener al menos un usuario \textbf{administrador} que podrá acceder a los datos
    almacenados referentes únicamente a su negocio. Estos usuarios serán los trabajadores del negocio o el propio
    dueño.

    \item Existirá un \textbf{superadministrador} que acceda al total de los datos del aplicativo
    para su gestión y control de errores.
\end{itemize}

En este tipo de administración se podrán crear funcionalidades al gusto para hacer correcciones o
cualquier tipo de necesidad, siendo también posible la accesibilidad por roles de usuario.
Los clientes solo podrán acceder a sus datos desde los diferentes sitios webs creados.