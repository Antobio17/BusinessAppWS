\chapter{Estado del arte}

El software libre y sus licencias \cite{gplv3} ha permitido llevar a cabo una expansión del
aprendizaje de la informática sin precedentes.

\section{Aplicaciones actuales}

En la actualidad existen diversas páginas para la reserva de citas en distintos negocios de cita previa:

\vspace{-0.5em}
\begin{itemize}
    \item \textbf{miCita}: Aplicación móvil desarrollada con el propósito de gestionar el apartado de citas de un
    negocio o empresa. En ella se rellenaran los datos del propio negocio con un asistente de configuración donde
    se completará información como los servicios ofrecidos y horarios. Los clientes podrán reservar su cita y
    observar el calendario en tiempo real desde la aplicación. Para acceder a está funcionalidades como cliente
    se debera abonar una cuota mensual de \textbf{19,90\euro/mes (IVA no incluido)}.\cite{micita}

    \item \textbf{Reservio}: Aplicación web para la gestión de reservas o citas en un negocio.
    Con un calendario para la reservas se podrá acceder como trabajador o como cliente permitiendo la reserva online.
    Además tendrá la posibilidad de gestionar los perfiles de los clientes  y estos podrán ser recompensados
    con pases en bas a la fidelidad. Esta aplicación también incluye asistencia de negocio con estadísticas y
    recordatorios via SMS.\cite{reservio}

    \item \textbf{Barberius}: Barberius es una tienda online de venta de productos de barbería.
    Dispone de un gran catálogo de productos a los que se podrán aplicar filtros a nuestras búsquedas
    para encontrar fácilmente lo que necesitemos. Se podrán hacer pedidos de los productos seleccionados
    teniendo un seguimiento de nuestras compras gracias al inicio de sesión con un perfil personal.\cite{barberius}
\end{itemize}

\subsection{Crítica al estado del arte}

Las aplicaciones mencionadas en el apartado anterior son los subconjuntos necesario para la creación de este proyecto.
Las dos primeras incluyen la gestión de citas, y usuario (en el caso de \textit{Reservio}). La tercera web constituye
la parte de la venta online, donde se gestiona todo lo relacionado con pedidos, pagos, envíos, etc:

\vspace{-0.5em}

\begin{itemize}
    \item \textbf{miCita}: un punto negativo al usar una aplicación como esta es que no tienes una interfaz propia
    que te diferencie de otras tiendas que usen esta mismo medio.

    \item \textbf{Reservio}: .

    \item \textbf{Barberius}: .
\end{itemize}

\section{Propuesta}
Teniendo en cuenta la situación actual sobre el estado del arte en lo referente a aplicaciones para
gestión de negocios de cita previa, se propone crear un sistema genérico que permita la administración
de estos. La principal finalidad del proyecto es conseguir un sistema sólido y unificado, en el que se
puedan integrar todos los negocios que se deseen (de características similares) de una manera sencilla.

No obstante al hacer una separación entre Front-End y Back-End será posible hacer única cada web si se
desea aunque internamente estén montadas con la misma estructura de datos.

Además de la gestión de citas se incluirán las funcionalidades necesarias para la implementación de una
tienda online en el que cada negocio podrá vender sus productos sin necesidad de atarse únicamente
a sus clientes locales.

Para la parte del Back-End se creara una vista de administración con roles de usuario:

\vspace{-0.5em}
\begin{itemize}
    \item Cada negocio deberá tener un usuario \textbf{administrador} que podrá acceder a los datos
    almacenados referentes únicamente a su negocio.

    \item Existirá un \textbf{super administrador} que acceda al total de los datos del aplicativo
    para su gestión y control de errores.
\end{itemize}

En este tipo de administración se podrán crear funcionalidades al gusto para hacer correcciones o
cualquier tipo de necesidad, siendo también posible la accesibilidad por roles de usuario.
Los clientes solo podrán acceder a sus datos desde los diferentes sitios webs creados.