\thispagestyle{empty}

\begin{center}
{\large\bfseries Herramienta de gestión de negocios de cita previa y venta de productos: \\
Web Service de gestión de negocios de cita previa y venta de productos junto con
aplicación web/móvil consumidora del mismo }\\
\end{center}
\begin{center}
Antonio Jiménez Rodríguez\\
\end{center}

%\vspace{0.7cm}

\vspace{0.5cm}
\noindent{\textbf{Palabras clave}: \textit{desarrollo web}, \textit{aplicación móvil}, \textit{APIRest},
\textit{código abierto}}
\vspace{0.7cm}

\noindent{\textbf{Resumen}\\
	
En el presente Trabajo de Fin de Grado se expondrá el proceso de creación de un Web Service donde será posible gestionar
múltiples negocios de cita previa a la vez, con una administración para su monitorización y que pueda utilizarse
como puente hacia la Base de Datos. También se desarrollará una aplicación web/móvil de un negocio real donde se
podrá ver el acoplamiento Front-End y Back-End.

El desarrollo del proyecto constará de distintas fases, análisis del problema, análisis de requisitos,
diseño del modelo de datos, diseño de interfaz de usuario y finalmente la implementación de todos estos estudios.

El diseño de la interfaz tiene gran importancia ya que deberá ser intuitiva y llamativa para el usuario.
La zona de tienda deberá ser cómoda de usar para mejorar la experiencia y aumentar las probabilidades de
que se efectúen compras. Así mismo el apartado para las reservas de citas deberá ser sencilla,
con el objetivo de facilitar la reserva de citas quitándole la necesidad de  llamar por teléfono para
saber que días hay disponibles. Esto repercute directamente en los trabajadores del negocio ya que no
tendrán la necesidad de atender a esos clientes y podrán seguir centrados en su trabajo.

Por otra parte el diseño del modelo de datos y la administración de la misma también tiene un papel importante.
En esta deberá haber roles de usuarios donde no todos las personas que accedan a ella puedan acceder a los mismos datos.
Por otra parte con un buen control de errores será posible llevar un a cabo una monitorización de todo
lo que pasa en las distintas webs y controlar/solucionar posibles fallos.

\cleardoublepage

\begin{center}
{\large\bfseries Business management tool for appointments and product sales: \\
Appointment and product sales business management web service together with web/mobile application consuming it }\\
\end{center}
\begin{center}
	Antonio Jiménez Rodríguez\\
\end{center}
\vspace{0.5cm}
\noindent{\textbf{Keywords}: \textit{web development}, \textit{app}, \textit{APIRest} , \textit{open source}
\vspace{0.7cm}

\noindent{\textbf{Abstract}\\



\cleardoublepage

\thispagestyle{empty}

\noindent\rule[-1ex]{\textwidth}{2pt}\\[4.5ex]

D. \textbf{Javier Martínez Baena}, Profesor del ...

\vspace{0.5cm}

\textbf{Informo:}

\vspace{0.5cm}

Que el presente trabajo, titulado \textit{\textbf{Herramienta de gestión de negocios de cita previa y venta de productos}},
ha sido realizado bajo mi supervisión por \textbf{Antonio Jiménez Rodríguez},
y autorizo la defensa de dicho trabajo ante el tribunal que corresponda.

\vspace{0.5cm}

Y para que conste, expiden y firman el presente informe en Granada a Junio de 2022.

\vspace{1cm}

\textbf{El director: }

\vspace{5cm}

\noindent \textbf{(Javier Martínez Baena)}

\chapter*{Agradecimientos}




