\thispagestyle{empty}

\begin{center}
{\large\bfseries Herramienta de gestión de negocios de cita previa y venta de productos: \\
Web Service de gestión de negocios de cita previa y venta de productos junto con
aplicación web/móvil consumidora del mismo }\\
\end{center}
\begin{center}
Antonio Jiménez Rodríguez\\
\end{center}

%\vspace{0.7cm}

\vspace{0.5cm}
\noindent{\textbf{Palabras clave}: \textit{desarrollo web}, \textit{aplicación móvil}, \textit{APIRest},
\textit{código abierto}}
\vspace{0.7cm}

\noindent{\textbf{Resumen}\\
	
En el presente Trabajo de Fin de Grado se expondrá el proceso de creación de un Web Service donde será posible gestionar
múltiples negocios de cita previa a la vez, con una administración para su monitorización y que pueda utilizarse
como puente hacia la Base de Datos. También se desarrollará una aplicación web/móvil de un negocio real donde se
podrá ver el acoplamiento Front-End y Back-End.

El desarrollo del proyecto constará de distintas fases: análisis del problema, análisis de requisitos,
diseño del modelo de datos, diseño de interfaz de usuario y finalmente la implementación de todo ello.

El diseño de la interfaz tiene gran importancia ya que deberá ser intuitiva a la vez que llamativa para el usuario.
Si la zona de tienda es cómoda de usar mejorará la experiencia y aumentará las probabilidades de
que se efectúen compras. Así mismo el apartado para las reservas de citas deberá ser sencillo,
con el objetivo de facilitar la reserva de citas quitándole la necesidad al usuario de  llamar por teléfono para
saber que días hay disponibles. Esto repercute directamente en los trabajadores del negocio ya que no
tendrán la necesidad de atender a esos clientes y podrán seguir centrados en su trabajo.

Por otra parte, el diseño del modelo de datos y la administración de la misma también tiene un papel importante.
En esta deberá haber roles de usuarios donde no todas las personas con permisos puedan acceder a los mismos datos.
Además, con un buen control de errores será posible llevar a cabo una monitorización de todo
lo que pasa en las distintas webs y controlar/solucionar posibles fallos.

\cleardoublepage

\begin{center}
{\large\bfseries Business management tool for appointments and product sales: \\
Appointment and product sales business management web service together with web/mobile application consuming it }\\
\end{center}
\begin{center}
	Antonio Jiménez Rodríguez\\
\end{center}
\vspace{0.5cm}
\noindent{\textbf{Keywords}: \textit{web development}, \textit{app}, \textit{APIRest} , \textit{open source}
\vspace{0.7cm}

\noindent{\textbf{Abstract}\\

In this Final Degree Project, we will present the process of creating a Web Service where it will be possible to manage
multiple business appointments at the same time, with an administration for its monitoring and that can be used as a bridge to the database. We will also develop a web/mobile application of a real business where it will be possible to see the Front-End and Back-End coupling.

The development of the project will consist of different phases, problem analysis, requirements analysis,
data model design, user interface design and finally the implementation of all these studies.

The design of the interface is of great importance since it must be intuitive and eye-catching for the user.
The store area should be comfortable to use to enhance the experience and increase the likelihood of purchases are more likely to be made. Likewise, the section for booking appointments should be simple,
in order to facilitate appointment booking by removing the need to phone to find out what days are available. This has a direct impact on the business' employees, as they will no longer have to attend to these clients and will be able to continue focusing on their work.

On the other hand, the design of the data model and its administration also plays an important role.
In this, there should be user roles where not all the people who access it can access the same data.
On the other hand, with a good error control it will be possible to carry out a monitoring of everything that happens on the different websites and control/solve possible failures.


\cleardoublepage

\thispagestyle{empty}

\noindent\rule[-1ex]{\textwidth}{2pt}\\[4.5ex]

D. \textbf{Javier Martínez Baena}, Profesor del Departamento de Ciencias de la Computación e Inteligencia Artificial de la Universidad de Granada.

\vspace{0.5cm}

\textbf{Informo:}

\vspace{0.5cm}

Que el presente trabajo, titulado \textit{\textbf{Herramienta de gestión de negocios de cita previa y venta de productos}},
ha sido realizado bajo mi supervisión por \textbf{Antonio Jiménez Rodríguez},
y autorizo la defensa de dicho trabajo ante el tribunal que corresponda.

\vspace{0.5cm}

Y para que conste, expiden y firman el presente informe en Granada a Junio de 2022.

\vspace{1cm}

\textbf{El director: }

\vspace{5cm}

\noindent \textbf{(Javier Martínez Baena)}

\chapter*{Agradecimientos}

A mis compañeros de trabajo por hacerme crecer como profesional y persona.

A mi familia y amigos por apoyarme siempre en esta etapa y creer en mi.

A mi tutor, D. Javier Martínez Baena por su ayuda, predisposición y compresión en el camino.


